\documentclass[oribibl]{llncs}

%\usepackage{pslatex}

\usepackage{times}
\usepackage{amsmath}
\usepackage{amsfonts}
\usepackage{amssymb}
\usepackage{mathpartir} % ./
\usepackage{color}
\usepackage{xspace} 
\usepackage[pdftex]{graphicx}
%\usepackage[small]{caption}
\usepackage{fancybox}
\usepackage{url}
\usepackage{multirow}
\usepackage{comment}
\usepackage{listing}
\usepackage{colortbl}

%\usepackage{times}
\usepackage{framed}
\usepackage{lineno}

%!TEX root =  main.tex

\newcommand{\cvc}{\textsc{cvc}{\small 4}\xspace}
\newcommand{\cvciii}{\textsc{cvc}{\small 3}\xspace}
\newcommand{\ziii}{\textsc{z}{\small 3}\xspace}
\newcommand{\teq}{\approx}
\newcommand{\cc}[1]{#1^*}
\newcommand{\M}{\mathcal{M}}
\def\AIF{\qtab\keyword{if}\ }
\def\THEN{\ \keyword{then}\ }
\def\AELSE{\untab\qtab\keyword{else}\ }
\def\FI{\untab}
\def\RETURN{\keyword{return}\ }
\def\ENDPROC{\untab}
%\newtheorem{remark}{Remark}


\newcommand{\define}[1]{\textsl{#1}}
\newcommand{\defas}{\stackrel{\mathrm{def}}{=}}

\newcommand{\rem}[1]{\textcolor{magenta}{[#1]}}
\newcommand{\remvk}[1]{\textcolor{red}{[#1]}}
\newcommand{\ct}[1]{\rem{#1 --ct}}
\newcommand{\ajr}[1]{\rem{#1 --ajr}}
\newcommand{\vk}[1]{\remvk{#1 --vk}}

\newcommand{\terms}{\mathbf{T}}
\newcommand{\sorts}{\mathbf{S}}
\newcommand{\vals}{\mathbf{V}}
\newcommand{\vars}{\mathbf{X}}
%\newcommand{\sorts}{\mathsf{sort}}
\newcommand{\funcs}{\mathsf{func}}

\newcommand{\I}{\mathcal{I}}
\newcommand{\J}{\mathcal{J}}
\newcommand{\Iu}{I}
\newcommand{\mods}{\mathbf{I}}
\newcommand{\lan}{\mathbf{L}}
\newcommand{\qlan}[1]{\mathcal{Q}( #1 )}
\newcommand{\props}{\mathbf{P}}

\newcommand{\modsof}[2]{{\llbracket {#1} \rrbracket}_{#2}}

\newcommand{\ent}[1][]{\models_{#1}}
\newcommand{\tent}{\ent[T]}

\newcommand{\ssorts}[1]{#1^\mathrm{s}}
\newcommand{\sfuns}[1]{#1^\mathrm{f}}

\newcommand{\con}[1]{\mathsf{#1}}
\newcommand{\Bool}{\con{Bool}}
\newcommand{\Int}{\con{Int}}
\newcommand{\Real}{\con{Real}}
\newcommand{\ite}{\con{ite}}
\newcommand{\ev}{\con{ev}}
\newcommand{\size}{\con{size}}

\newcommand{\TD}{T_\mathrm{D}}

\newcommand{\ltrue}{\top}
\newcommand{\lfalse}{\bot}

\newcommand{\euf}{\ensuremath{\mathrm{EUF}}\xspace}
\newcommand{\tra}{\ensuremath{\mathrm{RA}}\xspace}
\newcommand{\tia}{\ensuremath{\mathrm{IA}}\xspace}
\newcommand{\lra}{\ensuremath{\mathrm{LRA}}\xspace}
\newcommand{\lia}{\ensuremath{\mathrm{LIA}}\xspace}
\newcommand{\lira}{\ensuremath{\mathrm{LIRA}}\xspace}
\newcommand{\uflia}{\ensuremath{\mathrm{UFLIA}}\xspace}
\newcommand{\larel}{\bowtie}

\newcommand{\nmf}[1]{#1\negthinspace\downarrow}
\newcommand{\interp}[1]{[ \negthinspace [ #1 ] \negthinspace ]}

\newtheorem{thm}{Theorem}
\newtheorem{cor}{Corollary}
\newtheorem{lem}{Lemma}
\newtheorem{defn}{Definition}
%\newtheorem{claim}{Claim}

\newcommand{\opdivd}{\small\mathsf{div}}
\newcommand{\opdivl}{\small\mathsf{div}^{-}}
\newcommand{\opdivu}{\small\mathsf{div}^{+}}
\newcommand{\opmod}{\xspace \mathsf{mod} \xspace}
\newcommand{\oplcm}{\xspace \mathsf{lcm} \xspace}
\newcommand{\opdivides}{\xspace \mid \xspace}
\newcommand{\opliapol}{b}
\newcommand{\optoint}{\small\mathsf{to\_int}}

\newcommand{\funcsolve}{\small\mathsf{solve}}
\newcommand{\funcsmtsolve}{\small\mathsf{DPLL_{\forall T}}}
\newcommand{\funcqi}{\small\mathsf{check}_\forall}

\newcommand{\Set}[1]{\left\{#1\right\}}
\newcommand{\parens}[1]{\left(#1\right)}
\newcommand{\tuple}[1]{\left\langle#1\right\rangle}

\newcommand{\bigor}{\bigvee}
\newcommand{\bigand}{\bigand}

\newcommand{\divides}{\mid}
\newcommand{\notdivides}{\nmid}
\newcommand{\fvars}{FV}

\newcommand{\normalize}[1]{{#1}\!\!\downarrow}
\newcommand{\intnormalize}[1]{{#1}\!\!\downarrow_{\Int}}
\newcommand{\unnormalize}[2]{{#1}\!\!\updownarrow^{#2}}
\newcommand{\approxg}[2]{{#1}\!\!\mid_{#2}}

\newcommand{\transform}{\ \leadsto \ }

\newcommand{\purify}[1]{\lfloor #1 \rfloor}
\newcommand{\purifyrec}[1]{\purify{#1}^\ast}
\newcommand{\purifyg}[3]{\purify{#1}_{(#2,#3)}}
\newcommand{\unpurify}[1]{\lceil #1 \rceil}
\newcommand{\quants}[1]{\mathcal{Q}(#1)}
\newcommand{\guardspos}[1]{\mathcal{A}(#1)}
\newcommand{\guardsneg}[1]{\mathcal{B}(#1)}
\newcommand{\activequants}[2]{\mathrm{Q}(#1,#2)}
\newcommand{\lits}{\#\mathsf{lits}}
\newcommand{\opmax}{\mathsf{max}}
\newcommand{\opmin}{\mathsf{min}}

\renewcommand{\shadowsize}{1pt}

\begin{document}

\title{Conflicts, Models and Heuristics for Quantifier Instantiation in SMT}

\author {Andrew Reynolds\inst{1}}

\institute{
Department of Computer Science, The University of Iowa, USA
}
\maketitle

\pagestyle{plain}
%\pagestyle{empty}

\begin{abstract}
[SMT solvers]
[Quantifier Instantiation]
[Techniques for UF]
[New techniques for theories]
[Summary]
\end{abstract}

\section{Introduction}

[SMT solvers now handle quantifiers]

[Important in Applications]

The use of quantified formulas is highly important in a number of applications,
including automated theorem proving, software verification, synthesis, and planning.

[Challenging in Theories]

[Handled well in Practice]
In spite of the theoretical challenges, many uses of quantified formulas
can be handled well in practice by modern automated theorem provers and SMT solvers.

[Landscape]

[Overview]

\section{Support for Quantified Formulas in an SMT Solver}

Most modern SMT solvers are based on the DPLL($T$) solving architecture,
where a set of decision procedures for $T$ are modularly combined with a SAT solver for propositional satisfiability~\cite{}.
Given a set of $T$-formulas $\Gamma$ as input to a DPLL($T$)-based SMT solver, its underlying SAT solver
abstracts each $T$-literal in $\Gamma$ as unique Boolean variable.
The SAT solver clausifies $\Gamma$ and returns that 
either $\Gamma$ is propositionally unsatisfiable (in which case it is also $T$-unsatisfiable),
or returns a set of literals $M$ that propositionally entail it.
We write $M \models_p \Gamma$ to denote this case, and refer to $M$ as a satisfying assignment for $\Gamma$.
It then uses a decision procedure for $T$ to
check the $T$-satisfiability of $M$, that is, whether there exists of model of $M$ that is consistent according to theory $T$.
If $M$ is $T$-satisfiable, then we may conclude that $\Gamma$ is $T$-satisfiable.
Otherwise, it returns a subset $C$ of $M$ that is $T$-unsatisfiable and adds the clause $\neg C$ to $\Gamma$,
which is often referred to as a conflict clause.

%In DPLL($T$), each theory literal and closed universally quantified formula is abstracted as a unique Boolean variable.
%The underlying SAT solver 

DPLL($T$)-based SMT solvers have been extended in the past decade with approaches for univeral and existential quantification~\cite{}.
For consistency, in this paper we assume all existential quantification is rewritten to universal quantification by the rewrite:
\[
\exists \vec x\, P( \vec x ) \transform \neg \forall \vec x\, \neg P( \vec x )
\]
In a DPLL($T$)-based approach for quantified formulas, 
each closed (universally) quantified formula is abstracted as a Boolean variable by the SAT solver as well.

\begin{example}
Consider the set of $UFLIA$-formulas:

\end{example}


\begin{comment}
\begin{enumerate}
\item If $\Gamma$ is propositionally unsatisfiable
\begin{enumerate}
\item[\ ] return ``unsat".
\end{enumerate}
\item[\ ] Otherwise, let $M$ be a set of literals such that $M \models_p \Gamma$.
\item[\ ] If $M$ is $T$-unsatisfiable
\begin{enumerate}
\item[\ ] return $\funcsmtsolve( \Gamma \cup \neg C )$ for some $T$-unsatisfiable $C \subseteq M$.
\end{enumerate}
\item[\ ] Otherwise, let $( r, L ) = \funcqi( M \backslash \quants{M}, \quants{M} )$.
\item[\ ] If $r$ is ``unknown"
\begin{enumerate}
\item[\ ] return $\funcsmtsolve( \Gamma \cup L )$.
\end{enumerate}
\item Otherwise, return ``sat".
\end{enumerate}
\end{comment}



\begin{figure}[t]
\begin{framed}
\begin{internallinenumbers}
$\funcsmtsolve( \Gamma )$:
\begin{enumerate}
\item[\ ] If $\Gamma$ is propositionally unsatisfiable,
\begin{enumerate}
\item[\ ] return ``unsat".
\end{enumerate}
\item[\ ] Otherwise, let $M = E \uplus Q$ be a set of literals such that $M \models_p \Gamma$.
\item[\ ] If $E$ is $T$-unsatisfiable,
\begin{enumerate}
\item[\ ] return $\funcsmtsolve( \Gamma \cup \neg C )$ for some $T$-unsatisfiable $C \subseteq E$.
\end{enumerate}
\item[\ ] Otherwise, let $( r, L ) = \funcqi( E, Q )$.
\item[\ ] If $r$ is ``unknown",
\begin{enumerate}
\item[\ ] return $\funcsmtsolve( \Gamma \cup L )$.
\end{enumerate}
\item[\ ] Otherwise, return ``sat".
\end{enumerate}
$\funcqi( E, Q )$:
\begin{enumerate}
\item[\ ] Do one of the following:
\begin{enumerate}
\item[\ ] Return $( \text{``sat"}, \emptyset )$, if $E \cup Q$ is $T$-satisfiable.
\item[\ ] Return $( \text{``unknown"}, L )$ for some set $L$ of $T$-lemmas.
\end{enumerate}
\end{enumerate}
\end{internallinenumbers}
\end{framed}
\vspace{-2ex}
\caption{An abstract procedure for $T$-inputs $\Gamma_0$ with quantified formulas in an SMT solver.
In Line 5, $E$ is quantifier-free, and the atoms of each literal in $Q$ are universally quantified formulas.
\label{fig:smtq}}
\end{figure}

Figure~\ref{fig:smtq} gives an abstract procedure $\funcsmtsolve$
for establishing the $T$-satisfiability of a set of $T$-formulas $\Gamma$ possibly containing universal quantification.
Lines 2 through 6 correspond to a DPLL($T$)-based approach for quantifier-free inputs.
In Line 5, we partition $M$ into two parts $E$ and $Q$,
where $E$ is quantifier-free and the atoms of literals in $Q$ are universally quantified formulas.
If the quantifier-free portion is $T$-satisfiable,
we proceed to Line 7, which invokes the procedure $\funcqi$ on $E$ and $Q$.
Abstractly, this procedure either may determine that $E \cup Q$ is $T$-satisfiable and return the pair $( \text{``sat"}, \emptyset )$,
or otherwise will return ( ``unknown", $L$ ), where $L$ is a set of formulas that are valid in theory $T$,
after which the procedure $\funcsmtsolve$ either terminates with ``sat" or adds $L$ to $\Gamma$ and repeats.

Designing support for quantified formulas in DPLL($T$)-based SMT solvers 
thus depends on how the function $\funcsmtsolve$ is implemented.
When doing so, recurrent questions include:
\begin{itemize}
\item How can we establish that $E \cup Q$ is $T$-satisfiable (when $Q$ is non-empty)?
\item What lemmas $L$ should we return?
\end{itemize}
The remainder of this paper will focus on these questions.

It is important to note that some SMT approaches to quantified formulas~\cite{}
reason about quantified formulas \emph{eagerly} during the DPLL($T$) search.
In terms of Figure~\ref{fig:smtq}, these approaches invoke $\funcqi$
for sets $M$ that are incomplete and do not necessarily propositinally entail all formulas in $\Gamma$.
The advantage of doing so is that lemmas returned by $\funcqi$ may help prune the search,
while the disadvantage is that calling $\funcqi$ may be expensive and lead to non-termination.
For simplicity, we assume a lazy approach for handling quantified formulas in this paper.

\subsection{Skolemization and Instantiation}


\section{E-matching}

[Introduction]

[Example]

[Extension to EUF]

[Intuition]

\subsection{Pattern Selection}

\subsection{Challenge: Too Many Instances}

[Data]

[Non-termination]

\subsection{Challenge: Incompleteness}

[Example]

[Pencil-and-paper]

\section{Conflict-Based Instantiation}

[Introduction]

[Example]

[Extension to EUF]

[Intuition]

\subsection{Impact}

\subsection{Challenge: Finding Conflicting Instances}

\section{Model-Based Instantiation}

[Introduction]

[Example]

\subsection{Impact}

\subsection{Completeness}

[Fixed-finite, finite, finite instantiation]

\subsection{Challenge: Constructing Models}

[Example]

\section{Counterexample-Guided Instantiation}

[Restate challenges]

[Quantifier Elimination]

[Basic idea]

[Example]

[Results]

\section{Summary}

[for UF]

[for no UF]

\section{Future Work}

[Engineering existing]

[Counterexample-guided for theories]

[Combining E-matching + counterexample-guided]

\bibliographystyle{abbrv}
\bibliography{main}


\end{document}






